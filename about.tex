\chapter{About the course}

\section{Course Data}

Code: 21754

Course name: ``Xarxes de Sensors Sense Fils''

Teacher: Luis Sanabria and Jaume Barcelo

Credits: 4

Year: 3rd or 4th year (optional)

Trimester: Spring

\section{Introduction}
The reduction in price and size of computing and wireless communication platforms over the last years opens a new possibility for gathering and processing information: Wireless Sensor Networks.
A wireless sensor node is a devices of small dimensions with wireless communication capabilities.
In wireless sensor nodes, the communication is often established with other wireless sensor nodes to exchange or pass information.
It is common to have this data directed to an special device that gathers all the data and is called the network sink.
As wireless sensor nodes are often battery-powered, energy saving is a relevant issue in these networks.

\section{Syllabus}
\begin{itemize}
  \item Lectures
  \begin{enumerate}
    \item Introduction to WSNs
    \item Arduino Platform
    \item Zigbee and 802.15.4
    \item Processing
    \item tbd
    \item tbd
    \item tbd
    \item Project presentation
    \item Invited talk
  \end{enumerate}
  \item Seminars
  \begin{enumerate}
    \item Dimming LED + button
    \item tbd
    \item tbd
    \item tbd
  \end{enumerate}
\item Lab Assignments
  \begin{enumerate}
    \item Project proposal and planning
    \item Project implementation
    \item Project prototype (alpha)
    \item Project prototype (beta)
    \item Project demonstration
  \end{enumerate}
\end{itemize}

\section{Bibliography}

The course closely follows the book:

Robert Faludi ``Building Wireless Sensor Networks'' (\cite{faludi2010bws}).

\section{Evaluation Criteria}

The grading is distributed as follows:
\begin{itemize}
\item Lectures continuous assessment, 20\%
\item Lab assignments, 20\%
\item Project proposal, 20\%
\item Project presentation, 20\%
\item Project demonstration, 20\%
\end{itemize}

It is necessary to obtain a decent mark in all the different evaluation aspects.
To pass the course, 50 out of the total 100 points need to be obtained.

\section{Survival guide}

\subsection{Questions and doubts}
WE like to receive questions and comments.
Normally, the best moment to express a doubt is during the class, as it is likely that many people in the class share the same doubt.
If you feel that you have a question that needs to be discussed privately, we can discuss it right after the class.

\subsection{Continuous feedback}
At the end of the lecture, we will ask you to anonymously provide some feedback on the course. 
In particular, I always want to know:
\begin{itemize}
\item What is the most interesting thing we have seen in class.
\item What is the most confusing thing in the class.
\item Any other comment you may want to add.

In labs, I will ask each group to hand in a short (few paragraphs) description of the work carried out in class, and the members of the group that have attended the class.
Note that this is different from the deliverables, which are the ones that are actually graded.
\end{itemize}

\subsection{How to make you teachers happy}

Avoid speaking while we are talking.
