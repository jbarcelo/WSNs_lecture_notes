\chapter{Practice: Installing the Arduino IDE}\label{pract:settingTheIDE}
\section*{Suggested read: Chaper~ \ref{introToArduino}}

In the following practice, you will spend some time getting to know the Arduino platform, its connections and how to interact with it through a PC.

\section{Reviewing the hardware}\label{pract:settingTheIDE:hardware}
As you were able to see in Figure~\ref{fig:Arduino_schematics}, the Arduino board contains a whole computer on a small chip, although it is at least a thousand times less powerful.

Taking a closer look at Figure~\ref{fig:Arduino_schematics}, you will be able to see \emph{14 Digital IO pins (pins 0-13), 6 Analogue IN pins (0-5) and 6 Analogue OUT pins (pins 3, 5, 6, 9, 10, and 11)}.

The \emph{Digital IO} pins, as the name suggests can be set to input or output. Their function is specified by the sketch you create in the IDE (more on IDE in Section~\ref{pract:settingTheIDE:IDE}). The \emph{Analogue IN} ports take analogue values (i.e., voltage readings from a sensor) and convert them into a number between 0 and 1023. As for the \emph{Analogue OUT} ports, are actually digital pins that can be reprogrammed for analogue output using the sketch you can create in the IDE.

\section{The Arduino IDE}\label{pract:settingTheIDE:IDE}
The Arduino Integrated Development Environment (IDE) is the responsible for making your code work in the Arduino board. Without entering in much unnecessary detail, what the IDE does is to translate your code into C language and compile it using \texttt{avr-gcc}, which makes it understandable to the micro-controller. This last step hides away as much as possible the complexities of programming micro-controllers, so you can spend more time thinking on your actual code.

You can download the Arduino IDE~\emph{\color{blue}{\href{http://www.arduino.cc/en/Main/Software}{from here}}}. If you are using \emph{Linux} or \emph{Windows} operating systems, just double click the downloaded file. This will open a folder named \emph{arduino-[version]}, such as \emph{arduino-1.0}. Place the folder wherever you want in your system. On the Mac, just double click the downloaded file, this will open a disk image containing the Arduino application. Drag a drop the application icon to your Applications folder.

Do not open your installed application yet. First you must teach your computer to detect the Arduino hardware through the USB ports.

\subsection{Configuring the USB ports for detecting the Arduino}
In Linux and OS X, the USB controllers are the same used by the operating system.

\emph{\bf{On the Mac}}, plug the Arduino into an USB port.

The PWR light on the board should come on. Also, the LED labelled "L" should start blinking.

Then, a pop-up window telling you that a new network interface was found should appear. Proceed clicking "Network Preferences...", and then "Apply". Although it may appear with a status of "Not Configured", the Arduino is ready for work.

\emph{\bf{Windows}} machines, plug your Arduino and the ``Found new Hardware Wizard'' will appear. After the wizard tries to find the driver on the Internet, you will be able to select "Install from a list or specific location" button. Choose it and click next. You will be able to find the drivers under the "Drivers" folder of the Arduino Software download.

Once the drivers are installed, you can launch the IDE and start using Arduino.

\subsection{Identifying the port connected to the Arduino}
In the case of the \emph{\bf{Mac}}, once in the Arduino IDE, select "Serial Port" from the "Tools" menu. Select \texttt{/dev/cu/.usbmodem}; this is the name that your computer uses to refer to the Arduino board.

For \emph{\bf{Windows}}, under the operating system "Start" menu open the "Device Manager" by right-clicking on "Computer" (Vista) or "My Computer" (XP), then choose properties. On XP, click "Hardware" and choose Device Manager. On Vista, click "Device Manager".

Look for the Arduino device in the list under "Ports (COM \& LPT)". Your device name will be followed by a port number, usually "COM\#", where \# refers to a number.

Once you have identified the COM port number for the Arduino connection, you can select that port from the Tools~$>$~Serial Port menu in the Arduino IDE.

Now the Arduino IDE can talk with the Arduino board and program it.

\subsection{What's the deal with Linux users?}
As mentioned before, IDE uses the same USB controllers than Linux. So, in order to effectively detect your Arduino in Linux, simply connect it to your PC, open a Terminal a type \texttt{ls dev/ttyUSB*}. This will display all available ports. Your Arduino serial port will probable be something like \texttt{/dev/ttyUSB0}.
