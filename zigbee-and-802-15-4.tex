\chapter{An introduction to Zigbee and IEEE 802.15.4}

Zigbee and IEEE 802.15.4 are specification for the higher and lower network layers of WSNs.\footnote{This chapter is based on \cite{farahani2008zwn}}
Zigbee devices are designed for low-range, low-complexity, low-cost and low-consumption applications.
Zigbee includes low duty-cycle capabilities, which mean that the devices can sleep for most of the time.
A zigbee product can potentially run on batteries for years.
In our particular course, we use zigbee combined with Arduino.
As Arduino is not designed for low consumption, it kills the possibility of running our prototypes on batteries for such a long time.

Zigbee defines the network and application layers of the devices and relies on IEEE 802.15.4 for the MAC and PHY layers.
The specification of Zigbee standards is carried out by the Zigbee Alliance, which comprises all kind of companies (semiconductor industry, OEMs, software developers, etc.).

IEEE 802.15.4 operate in three different bands: 868 MHz, 915 MHz and 2.4 GHz.
The 2.4 GHz band is a band for industrial, scientific and medical uses (ISM).
This band is shared by many technologies, including the IEEE 802.11 family.
Spread spectrum techniques are used to alleviate the problem of interference.
The modulations that are used are BPSK, ASK and O-QPSK.

Direct sequence spread spectrum (DSSS) or parallel sequence spread spectrum (PSSS) is used.

\section{Devices}

There are two different types of devices: full-function devices (FFD) and reduced-function devices (RFD).
FFDs are more powerful and can assume any role in the network.
RFDs are simpler, and can assume only the role of ``Device'' in the IEEE 802.15.4 terminology (equivalent to ``ZigBee End Device'' in the ZigBee terminology.

The other two possible roles, that can be assumed only by FFDs are ``Coordinator'' and ``PAN Coordinator'' in the IEEE 802.15.4 terminology.
These are equivalent to ``ZigBee Router'' and  ``ZigBee Coordinator'' in the ZigBee terminology.

A ZigBee End Device can be the source or destination of a packet, but it cannot forward packets for other nodes.
A ZigBee Router can relay packets of other nodes.
And a ZigBee Coordinator is the head of the network.
In every every network, there is one and only one Coordinator.

Regarding the topologies, we can consider three different cases: star, tree and mesh.

\section{Channel Access}

In IEEE 802.15.4 there are two methods for networking: beacon-enabled networking and non-beacon networking.
In beacon-enabled networking the coordinator transmits periodical beacons to synchronize the network.
In these beacons, it is possible to define a super-frame structure.
The super-frame structure spans the channel time between two consecutive beacons, and differentiates three periods: Contention Access Period (CAP), Contention Free Period (CFP, optional) and Inactive Period (IP, optional).
The time is divided in slots, and multiple slots can be assigned to each of the different periods.
In the CAP, the devices use CSMA/CA to contend for channel access.
In the CFP, the channel time is reserved and only the devices that own the reservation can transmit.
In the IP, the devices can go to sleep to save energy.

In non-beacon networking, the devices must always contend for the channel.
The contention uses the CSMA/CA protocol.
Before transmitting, the devices sense the channel for the presence of an ongoing transmission.
If a transmission is detected, the devices backs off and re-attempts after a random period of time.

\section{MAC handshake}

In a beacon-enabled network, a node that wants to transmit data to the coordinator waits for the CFP if it has a reservation.
Otherwise, it waits for the CAP.
Then it transmits the packet (using CSMA/CA if it is in contention mode).
The sender can request acknowledgement and, in this case, the coordinator may acknowledge the reception.

When a coordinator wants to transmit data to a device it indicates in the beacon the destination of the packet.
The device processes the beacon and learns that there is data pending for it.
After that, the device sends a ``data request'' message to the coordinator that the coordinator must acknowledge.
Then the coordinator sends the data packet to the device which may acknowledge the correct reception.

In a non-beacon network, a device can transmit to the coordinator whenever the channel is sensed empty.
